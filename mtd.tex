\documentclass[aps, prc, reprint, amsmath]{revtex4-1}

\usepackage[bookmarksopen=true]{hyperref}
\usepackage{graphicx}
\graphicspath{{fig/}}


\newcommand{\widefig}[3][t]{
  \begin{figure*}[#1]
    \includegraphics{#2}
    \caption{\label{fig:#2}#3}
  \end{figure*}
}

\newcommand{\placeholderfig}[3][t]{
  \begin{figure}[#1]
    \centering
    \framebox{\parbox[c][.5\columnwidth]{\columnwidth}{
      placeholder
    }}
    \caption{\label{fig:#2}#3}
  \end{figure}
}

\newcommand{\avg}[1]{\langle #1 \rangle}
\newcommand{\nch}{N_\text{ch}}
\newcommand{\vnk}[2]{v_#1\{#2\}}


\begin{document}

\title{Extracting QGP properties via systematic model-to-data comparison}

\author{Jonah E.\ Bernhard}
\author{Steffen A.\ Bass}
\author{Christopher E.\ Coleman-Smith}
\author{Robert L.\ Wolpert}
\affiliation{Duke University}

\author{Snehalata Huzurbazar}
\affiliation{University of Wyoming}

\author{Peter Marcy}
\affiliation{Los Alamos National Lab}


\date{\today}

\begin{abstract}
  hello
\end{abstract}

\maketitle


\section{Introduction}

Similar to \cite{pratt-mtd, soltz-mtd}.


\section{Model}

\cite{bass-dumitru, nonaka-bass, song}
\cite{glauber}
\cite{kln}
\cite{vish}
\cite{cooper-frye}
\cite{iss}
\cite{urqmd1, urqmd2}
\cite{iebe}

\widefig{prior_draws_glb}{
  Prior model calculations using Glauber initial conditions.
  From left to right:
  average charged-particle multiplicity $\avg\nch$,
  elliptic flow two-particle cumulant $\vnk 2 2$,
  and triangular flow two-particle cumulant $\vnk 3 2$.
  Each plot has 254 lines corresponding to the 254 design points.
  Data points are experimental measurements from ALICE \cite{alice-cumulants}.
}

\widefig{prior_draws_kln}{
  Same as FIG.~\ref{fig:prior_draws_glb} for KLN initial conditions.
}


\section{Emulator}

\cite{gpml}
\cite{ohagan, higdon2008, higdon2014}

\placeholderfig{gp}{
  Conditioning a Gaussian process.
}

\placeholderfig{design}{
  Latin-hypercube experiment design.
}

\placeholderfig{pca}{
  Dimensionality reduction via principal component analysis.
}

\widefig{validation_glb}{
  Validation of the Gaussian process emulator for the Glauber model.
  Each plot shows emulator predictions against explicit calculations for the 64 validation design points and centrality bins 0--5\%, 20--25\%, and 40-45\%.
  The $x$-value of each data point is the emulator prediction with 95\% error bars; the $y$-value is the explicit calculation.
  The diagonal grey line represents $y = x$.
}


\section{Calibration}

\cite{alice-cumulants}
\cite{osu1, osu2}

\widefig{calibration_posterior_glb}{
  Posterior marginal and joint distributions of the calibration parameters for the Glauber model.
  On the diagonal are histograms of MCMC samples for the respective parameters,
  on the lower triangle are two-dimensional histograms of MCMC samples showing the correlation between pairs of parameters.
  The upper triangle is blank for now. \textbf{FIXME}
}

\widefig{calibration_posterior_kln}{
  Same as FIG.~\ref{fig:calibration_posterior_glb} for the KLN model.
}

\placeholderfig{posterior_comparison}{
  Comparison of posterior distributions from the Glauber and KLN models (boxplots or similar).
  Also comparison to prior guesses from OSU.
}

\begin{table}
  \caption{
    \label{tab:posterior}
    Quantitative summary of posterior distributions.
  }
  \begin{ruledtabular}
  \begin{tabular}{llll}
    Parameter & Mean & Median & Confidence intervals \\
    $\eta/s$ & & & \\
  \end{tabular}
  \end{ruledtabular}
\end{table}

\placeholderfig{chi_squared}{
  Goodness of fit comparison.
}

\widefig{posterior_draws_glb}{
  Random realizations of the calibrated posterior for the Glauber model.
  Similar to FIG.~\ref{fig:prior_draws_glb},
  except the lines are posterior emulator predictions instead of explicit prior calculations.
  The red line is the maximum a posteriori point of the MCMC chain.
}

\widefig{posterior_draws_kln}{
  Same as FIG.~\ref{fig:posterior_draws_glb} for the KLN model.
}



\section{Conclusion}



\appendix


\section{Emulator training}

\widefig{training_posterior_glb}{
  Posterior marginal and joint distributions of the Gaussian process hyperparameters for the first Glauber principal component.
  The notation $\ell\;x$ means the squared-exponential correlation length for parameter $x$. \textbf{IMPROVE THIS}
  On the diagonal are histograms of MCMC samples for the respective hyperparameters,
  on the lower triangle are two-dimensional histograms of MCMC samples showing the correlation between pairs of hyperparameters.
  The upper triangle is blank for now. \textbf{FIXME}
}

\widefig{training_posterior_kln}{
  Same as FIG.~\ref{fig:training_posterior_glb} for the KLN model.
}



\bibliography{mtd}


\end{document}
